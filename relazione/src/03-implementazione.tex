\section{Implementazione degli algoritmi}
\label{sec:impl}
\subsection{Elevamento a potenza in modulo}
\label{sec:impl_pot}
Operazioni come $b^e\bmod{k}$ vengono calcolate utilizzando un algoritmo noto con il nome di \textit{Right-to-Left binary exponentiation}\cite[par. 4.6.3]{Knuth}.

Non è possibile calcolare prima $b^e$ come verrebbe spontaneo, perché nel nostro caso sarebbe un numero troppo grosso da rappresentare. Per superare questo problema basta notare come:
$$c = (a \cdot b)\bmod{m}$$
sia uguale a
$$c = (a\bmod{m} \cdot b\bmod{m})\bmod{m}$$

\noindent Per cui ad ogni iterazione salveremo solo il modulo del risultato intermedio, e non tutto il prodotto.

Adesso per ottimizzare il numero di operazioni necessarie possiamo scrivere \textit{e} in notazione binaria:
$$e = \sum\limits_{i=0}^{n-1} a_i 2^i$$
Dove $a_i$ può essere 0 o 1, $a_{n-i} = 1$, e \textit{n} è il numero minimo di bit necessario per rappresentare \textit{e}.

Quindi:
$$b^e = \prod\limits_{i=0}^{n-1} (b^{2^i})^{a_i}$$

\noindent Per maggiore chiarezza, lo pseudocodice risultante è:

\begin{mdframed}[backgroundcolor=bg]
\begin{minted}[mathescape]{matlab}
function ModPow(b, e, k)
    if k = 1 then return 0
    result := 1
    b := b mod k
    while e > 0
        if (e mod 2 == 1):
            result := (result * b) mod k
        e := e / 2
        b := (b * b) mod k
    return result
\end{minted}
\end{mdframed}

\subsection{Altre scelte implementative}

La quantità di cifre calcolate a partire della posizione specificata, dipende dalla precisione della macchina utilizzata. In particolare Bailey stesso durante dei test ha ottenuto almeno nove cifre corrette usando aritmetica a 64 bit, mentre con usando 128 bit se ne hanno almeno ventiquattro.\cite[p.\ 4-5]{Bailey}
\newline

Nel programma presentato vengono utilizzati numeri in virgola mobile da 128 bit, implementati mediante una libreria standard.