\section{Test e verifica della correttezza}
Il programma è stato sviluppato seguendo il modello del \textit{Test Driven Development}, questo significa che sono stati realizzati i test per ogni singola funzione prima della funzione stessa.

Ogni file sorgente è accompagnato da un secondo file contenente i relativi test.

\noindent La maggior parte della suite di test utilizza la tecnica dell'oracolo, in pratica viene confrontato l'output della funzione implementata con risultati noti a priori.

Per esempio nel calcolo dell'elevamento a potenza in modulo vengono calcolati:
$$ 2^{100} \bmod 2 = 0 $$
$$ 50^{19800} \bmod 6 = 4 $$
$$ 367^{447} \bmod 739 = 687 $$

I risultati corretti sono stati trovati utilizzando il calcolatore online \textit{Wolfram Alpha}\footnote{Link: \href{http://www.wolframalpha.com/}{http://www.wolframalpha.com/}}.
\newline \newline
\noindent Per quanto riguarda invece il calcolo vero e proprio delle cifre di log(2) e $\pi$, oltre all'oracolo, sono stati effettuati anche test sull'\textit{overlapping} delle cifre.

Questo significa che vengono calcolate le prime otto cifre di $\pi$: 243F\underline{6A88}, e le cifre a partire dalla quinta: \underline{6A88}85A3. Si verifica poi che ci sia corrispondenza tra le quattro finali del primo calcolo e le quattro iniziali del secondo calcolo.

Più in generale, questo metodo è utilizzabile anche per capire quante cifre successive alla posizione desiderata sono corrette.
\newline \newline
\noindent Oltre a quelle già descritte, viene eseguita anche una semplice funzione per verificare di stare usando veramente numeri in virgola mobile da 128 bit. Il test consiste nel calcolare $\frac{1}{49} \cdot 49$ e verificare che il risultato sia 1.

Su sistemi a 64 bit fallirebbe poiché $\frac{1}{49}$ non è rappresentabile senza errore.