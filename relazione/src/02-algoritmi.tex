\section{Algoritmi}

\subsection{Il calcolo di log(2)}
Borwein e Plouffe osservarono come partendo dalla formula per il calcolo di $\log(2)$ potessero calcolarne le cifre binarie a partire da una posizione arbitraria.

\noindent Riprendendo quindi la formula scritta precedentemente:
$$\ln(2) = \sum\limits_{k=1}^{\infty} \frac{1}{2^k \cdot k}$$

\noindent Vogliamo calcolare le cifre \textit{binarie} che partono alla posizione $d+1$. Si può notare come questo sia equivalente a calcolare $\{2^d \cdot \log(2)\}$, dove \{\} indica la parte frazionaria\footnote{La definizione di parte frazionaria è: $\{x\} = x - \lfloor x \rfloor$.}.

Quindi possiamo scrivere:
\begin{equation*}
\begin{split}
\left\{2^d \cdot \log(2)\right\}
& = \left\{ \sum\limits_{k=1}^{\infty} \frac{2^{d-k}}{k} \right\} \\
& = \left\{ \left\{ \sum\limits_{k=1}^{d} \frac{2^{d-k}}{k} \right\} + \sum\limits_{k=d+1}^{\infty} \frac{2^{d-k}}{k} \right\} \\
& = \left\{ \left\{ \sum\limits_{k=1}^{d} \frac{2^{d-k}\bmod k}{k} \right\} + \sum\limits_{k=d+1}^{\infty} \frac{2^{d-k}}{k} \right\}
\end{split}
\end{equation*}

Nel primo passaggio la sommatoria è stata divisa in due sommatorie per separare la prima che possiede una parte intera, dalla seconda che invece è sicuramente minore di 1.\footnote{Questo è dato dal fatto che $2^{d-k} < 1$ per $k > d$.}

Nel secondo passaggio è stato aggiunto \textit{mod k}, giustificato dal fatto che ci interessa solo la parte frazionaria del quoziente della divisione per k.

Adesso si può notare come i termini della seconda sommatoria diventino piccoli molto in fretta, quindi a seconda della precisione di macchina disponibile, bastano poche iterazioni per un'approssimazione sufficiente a trovare la cifra in posizione d+1, ma anche qualcuna delle successive.
\bigbreak \noindent
Per l'implementazione si rimanda alla \hyperref[sec:impl]{\textit{Sezione 3}}.

\subsection{Il calcolo di Pi Greco}
Dopo essere riusciti con log(2), Borwein e Plouffe sono andati alla ricerca di formule analoghe che funzionassero con altre costanti. Ed è così che hanno trovato la formula per il calcolo di $\pi$:

\begin{equation*}
\begin{split}
\pi 
& = \sum\limits_{k=0}^{\infty} \left[ \frac{1}{16^k}\left( \frac{4}{8k+1} - \frac{2}{8k+4} - \frac{5}{8k+5} - \frac{1}{8k+6} \right) \right] \\
& =
4 \cdot S_1 - 2 \cdot S_4 - S_5 - S_6, \\
S_j & = \sum\limits_{k=0}^{\infty} \frac{1}{16^k \cdot (8k+j)}
\end{split}
\end{equation*}

Dove la sommatoria viene divisa in quattro sommatorie più piccole permettendo una notazione più compatta.
\bigbreak
Da qui si può fare un ragionamento molto simile al precedente: per calcolare la cifre \textit{esadecimali} che partono alla posizione $d+1$, occorre calcolare $\{16^d \cdot \pi\}$, ricordando che \{\} indica la parte frazionaria. Ovvero:
\begin{equation*}
\{16^d\pi\} =
\{4 \cdot \{16^dS_1\} - \{2 \cdot \{16^dS_4\}\} - \{16^dS_5\} - \{16^dS_6\}\}
\end{equation*}

Per calcolare in modo efficiente gli addendi di questa somma, applichiamo le stesse trasformazioni fatte nel paragrafo precedente:
\begin{equation*}
\begin{split}
\{16^dS_j\} & = \left\{\left\{\sum\limits_{k=0}^d \frac{16^{d-k}}{8k+j}\right\} + \sum\limits_{k=d+1}^\infty \frac{16^{d-k}}{8k+j}\right\} \\
& = \left\{\left\{\sum\limits_{k=0}^d \frac{16^{d-k}\bmod{8k+j}}{8k+j}\right\} + \sum\limits_{k=d+1}^\infty \frac{16^{d-k}}{8k+j}\right\}
\end{split}
\end{equation*}

Si nota subito la somiglianza con il risultato trovato per log(2), con la sola differenza che adesso andranno calcolati quattro diversi risultati per poi combinarli opportunamente insieme.

Il calcolo della prima sommatoria potrebbe sembrare costoso visto l'esponente molto grande al numeratore, ma è reso possibile dall'operazione di \textit{modulo}. Infatti viene utilizzato un algoritmo ottimizzato per occupare poco spazio, spiegato nella sezione \hyperref[sec:impl_pot]{\textit{Sezione 3.1}}.